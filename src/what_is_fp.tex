\begin{frame}
  What is Functional Programming?
\end{frame}

\begin{frame}
  Three fuzzy groups:

  \begin{itemize}
    \item<1-> \textbf{Procedural:}      Ada, BASIC, C
    \item<2-> \textbf{Object Oriented:} C\#, C++, Java
    \item<3-> \textbf{Functional:}      Lisp, F\#, Erlang, Scala, Haskell
  \end{itemize}

  % 3x to get through the list
  \pause
  \pause
  \pause

  Note: Focusing on 'pure' functional programming.
\end{frame}

\begin{frame}
  Procedural

  \begin{itemize}
    \item<1-> Modular sequences of steps
    \item<2-> Collections handled with loops
    \item<3-> Data exposed directly
    \item<4-> State transformed until correct
  \end{itemize}
\end{frame}

\begin{frame}[fragile]
  \begin{Verbatim}[commandchars=\\\{\}]
\PY{k+kt}{int} \PY{n+nf}{sum}\PY{p}{(}\PY{k+kt}{int} \PY{o}{*} \PY{n}{array}\PY{p}{,} \PY{k+kt}{int} \PY{n}{count}\PY{p}{)}
\PY{p}{\PYZob{}}
    \PY{k+kt}{int} \PY{n}{i}\PY{p}{;}
    \PY{k+kt}{int} \PY{n}{sum} \PY{o}{=} \PY{l+m+mi}{0}\PY{p}{;}

    \PY{k}{for} \PY{p}{(}\PY{n}{i} \PY{o}{=} \PY{l+m+mi}{0}\PY{p}{;} \PY{n}{i} \PY{o}{<} \PY{n}{count}\PY{p}{;} \PY{n}{i}\PY{o}{+}\PY{o}{+}\PY{p}{)}
    \PY{p}{\PYZob{}}
        \PY{n}{sum} \PY{o}{+}\PY{o}{=} \PY{n}{array}\PY{p}{[}\PY{n}{i}\PY{p}{]}\PY{p}{;}
    \PY{p}{\PYZcb{}}

    \PY{k}{return} \PY{n}{sum}\PY{p}{;}
\PY{p}{\PYZcb{}}
\end{Verbatim}

\end{frame}

\begin{frame}
  Object Oriented

  \begin{itemize}
    \item<1-> Objects
    \item<2-> Data hiding
    \item<3-> Types
    \item<4-> Collections handled with loop-like construct
  \end{itemize}
\end{frame}

\begin{frame}[fragile]
  \input{snippets/oo.tex}
\end{frame}

\begin{frame}
  Functional

  \begin{itemize}
    \item<1-> Computation is the evaluation of mathematical functions
    \item<2-> Free of side-effects
    \item<3-> No assignment
    \item<4-> No loops % consequence of no assigment
    \item<5-> Recursion!
  \end{itemize}
\end{frame}
